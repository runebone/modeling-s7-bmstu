\documentclass[a4paper,oneside,14pt]{extarticle}

\include{preamble}

\begin{document}

\include{title}
\setcounter{page}{2}
%\renewcommand{\contentsname}{СОДЕРЖАНИЕ}
%\tableofcontents

\section{Теоретическая часть}

\subsection{Равномерное распределение}

Случайная величина имеет равномерное распределение на отрезке $[a, b]$, если ее функция плотности распределения вероятностей
\[
f(x) =
\begin{cases}
    \begin{array}{c l}
        \dfrac{1}{b - a}, & a \leq x \leq b; \\
        0, & x < a \text{ или } x > b.
    \end{array}
\end{cases}
\]

Ее функция распределения в этом случае определяется выражением
\[
F(x) =
\begin{cases}
    \begin{array}{c l}
    0, & x < a; \\
    \dfrac{x - a}{b - a}, & a \leq x \leq b; \\
    1, & x > b.
    \end{array}
\end{cases}
\]

\subsection{Нормальное распределение}

Случайная величина имеет нормальное распределение, если ее функция плотности распределения вероятностей 
\[
    f(x) = \dfrac{1}{\sigma\sqrt{2\pi}}e^{\displaystyle -\frac{(x-m)^2}{2\sigma^2}} \hspace{0.5cm} (m \in \mathbb{R},\ \sigma > 0).
\]

Функция нормального распределения имеет вид
\[
    F(x) = \dfrac{1}{\sigma\sqrt{2\pi}} \int\displaylimits_{-\infty}^{x} e^{\displaystyle -\frac{(x - m)^2}{2\sigma^2}} dx.
\]

\subsection{Распределение Пуассона}

Дискретная случайная величина $X$ распределена по закону Пуассона, если она принимает целые неотрицательные значения с вероятностями
\[
    \mathrm{P}\{X = i\} = P(i;\, \lambda) = \dfrac{\lambda^i}{i!} e^{\displaystyle -\lambda}, \hspace{1cm} i = 0, 1, \dots
\]

Ее функция распределения
\[
    F(k;\, \lambda) = \mathrm{P}\{X \leq k\} = \sum\displaylimits_{i = 0}^k P(i;\, \lambda) = e^{\displaystyle -\lambda} \sum\displaylimits_{i = 0}^k \dfrac{\lambda^i}{i!}.
\]

\subsection{Распределение Эрланга}

Распределение Эрланга --- частный случай Гамма распределения, когда параметр $k$ является целым числом.

Случайная величина имеет Эрланговское распределение, если ее функция плотности распределения вероятностей 
\[
    f(x) =
    \begin{cases}
        \begin{array}{c l}
            x^{k-1}\dfrac{e^{-x/\theta}}{\theta^k \,(k-1)!}, & x \geq 0; \\
            0, & x < 0.
        \end{array}
    \end{cases}
\]

Функция распределения Эрланга
\[
    F(x; k, \theta) =
    \begin{cases}
        \begin{array}{c l}
            1 - e^{-x/\theta} \displaystyle \sum\displaylimits_{i = 0}^{k - 1} \dfrac{x^i}{\theta^i \,i!}, & x \geq 0; \\
            0, & x < 0.
        \end{array}
    \end{cases}
\]

\newpage

\section{Практическая часть}

\subsection{Равномерное распределение}

\begin{figure}[H]
	\centering
	\includegraphics[scale=0.6]{img/th_regular.pdf}
	\caption{Равномерное распределение --- функция распределения и функция плотности распределения вероятностей}
	\label{fig:}
\end{figure}

\begin{figure}[H]
	\centering
	\includegraphics[scale=0.6]{img/emp_regular.pdf}
	\caption{Равномерное распределение --- эмпирическая функция распределения и гистограмма}
	\label{fig:}
\end{figure}

\subsection{Нормальное распределение}

\begin{figure}[H]
	\centering
	\includegraphics[scale=0.7]{img/th_normal.pdf}
	\caption{Нормальное распределение --- функция распределения и функция плотности распределения вероятностей}
	\label{fig:}
\end{figure}

\begin{figure}[H]
	\centering
	\includegraphics[scale=0.7]{img/emp_normal.pdf}
	\caption{Нормальное распределение --- эмпирическая функция распределения и гистограмма}
	\label{fig:}
\end{figure}

\subsection{Распределение Пуассона}

\begin{figure}[H]
	\centering
	\includegraphics[scale=0.7]{img/th_poisson.pdf}
	\caption{Пуассоновское распределение --- функция распределения и функция плотности распределения вероятностей}
	\label{fig:}
\end{figure}

\begin{figure}[H]
	\centering
	\includegraphics[scale=0.7]{img/emp_poisson.pdf}
	\caption{Пуассоновское распределение --- эмпирическая функция распределения и гистограмма}
	\label{fig:}
\end{figure}

\subsection{Распределение Эрланга}

\begin{figure}[H]
	\centering
	\includegraphics[scale=0.7]{img/th_erlang.pdf}
	\caption{Эрланговское распределение --- функция распределения и функция плотности распределения вероятностей}
	\label{fig:}
\end{figure}

\begin{figure}[H]
	\centering
	\includegraphics[scale=0.7]{img/emp_erlang.pdf}
	\caption{Эрланговское распределение --- эмпирическая функция распределения и гистограмма}
	\label{fig:}
\end{figure}

\end{document}
